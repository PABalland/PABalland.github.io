% Options for packages loaded elsewhere
\PassOptionsToPackage{unicode}{hyperref}
\PassOptionsToPackage{hyphens}{url}
%
\documentclass[
]{article}
\usepackage{lmodern}
\usepackage{amssymb,amsmath}
\usepackage{ifxetex,ifluatex}
\ifnum 0\ifxetex 1\fi\ifluatex 1\fi=0 % if pdftex
  \usepackage[T1]{fontenc}
  \usepackage[utf8]{inputenc}
  \usepackage{textcomp} % provide euro and other symbols
\else % if luatex or xetex
  \usepackage{unicode-math}
  \defaultfontfeatures{Scale=MatchLowercase}
  \defaultfontfeatures[\rmfamily]{Ligatures=TeX,Scale=1}
\fi
% Use upquote if available, for straight quotes in verbatim environments
\IfFileExists{upquote.sty}{\usepackage{upquote}}{}
\IfFileExists{microtype.sty}{% use microtype if available
  \usepackage[]{microtype}
  \UseMicrotypeSet[protrusion]{basicmath} % disable protrusion for tt fonts
}{}
\makeatletter
\@ifundefined{KOMAClassName}{% if non-KOMA class
  \IfFileExists{parskip.sty}{%
    \usepackage{parskip}
  }{% else
    \setlength{\parindent}{0pt}
    \setlength{\parskip}{6pt plus 2pt minus 1pt}}
}{% if KOMA class
  \KOMAoptions{parskip=half}}
\makeatother
\usepackage{xcolor}
\IfFileExists{xurl.sty}{\usepackage{xurl}}{} % add URL line breaks if available
\IfFileExists{bookmark.sty}{\usepackage{bookmark}}{\usepackage{hyperref}}
\hypersetup{
  pdftitle={Economics of Networks - GEO3-3805},
  pdfauthor={Pierre-Alexandre Balland},
  hidelinks,
  pdfcreator={LaTeX via pandoc}}
\urlstyle{same} % disable monospaced font for URLs
\usepackage[margin=1in]{geometry}
\usepackage{longtable,booktabs}
% Correct order of tables after \paragraph or \subparagraph
\usepackage{etoolbox}
\makeatletter
\patchcmd\longtable{\par}{\if@noskipsec\mbox{}\fi\par}{}{}
\makeatother
% Allow footnotes in longtable head/foot
\IfFileExists{footnotehyper.sty}{\usepackage{footnotehyper}}{\usepackage{footnote}}
\makesavenoteenv{longtable}
\usepackage{graphicx,grffile}
\makeatletter
\def\maxwidth{\ifdim\Gin@nat@width>\linewidth\linewidth\else\Gin@nat@width\fi}
\def\maxheight{\ifdim\Gin@nat@height>\textheight\textheight\else\Gin@nat@height\fi}
\makeatother
% Scale images if necessary, so that they will not overflow the page
% margins by default, and it is still possible to overwrite the defaults
% using explicit options in \includegraphics[width, height, ...]{}
\setkeys{Gin}{width=\maxwidth,height=\maxheight,keepaspectratio}
% Set default figure placement to htbp
\makeatletter
\def\fps@figure{htbp}
\makeatother
\setlength{\emergencystretch}{3em} % prevent overfull lines
\providecommand{\tightlist}{%
  \setlength{\itemsep}{0pt}\setlength{\parskip}{0pt}}
\setcounter{secnumdepth}{-\maxdimen} % remove section numbering

\title{Economics of Networks - GEO3-3805}
\author{Pierre-Alexandre Balland}
\date{}

\begin{document}
\maketitle

\hypertarget{course-description}{%
\subsubsection{Course description}\label{course-description}}

Economists use concepts and methods from network science to understand a
complex, global, and interconnected world. In network thinking, the
fundamental unit of analysis consists of the relationships among
interacting units rather than the individual characteristics of these
units. The ability to analyze the particular way relationships are
organized - i.e.~the network structure - is crucial in understanding
complex phenomena in nature and society. In this course, we discuss how
a network perspective can help us to re-think key issues in economics.
More specifically, this course will introduce concepts and methods to
map and measure relationships and flows between people, firms, cities,
economic sectors, communities, or any other element of a complex
economic system. We will discuss how the structure of a system
influences its overall performance, why the relative position in a
network conditions the access to critical resources, and how
relationships are created and dissolved over time. This course consists
of lectures combined with computer exercises and online tutorials.

\hypertarget{what-you-will-learn}{%
\subsubsection{What you will learn}\label{what-you-will-learn}}

\begin{enumerate}
\def\labelenumi{\arabic{enumi}.}
\tightlist
\item
  What a network-based paradigm is, and how it can be applied to
  economics
\item
  How to identify, describe, and analyze the structure of networks
\item
  How the structure of economic systems influence their performance
\item
  How networks evolve in space and over time
\item
  Basic programming skills (in R) and advanced network analysis
  techniques
\end{enumerate}

\hypertarget{meet-the-instructors}{%
\subsubsection{Meet the instructors}\label{meet-the-instructors}}

\href{https://www.paballand.com/}{Pierre-Alexandre Balland} -
\href{mailto:p.balland@uu.nl}{\nolinkurl{p.balland@uu.nl}}\\
\href{n.a.debruin@uu.nl}{Anne de Bruin} -
\href{mailto:n.a.debruin@uu.nl}{\nolinkurl{n.a.debruin@uu.nl}}

\hypertarget{grading}{%
\subsubsection{Grading}\label{grading}}

The overall grade for the class will be based on an individual essay
(50\%) and a report of a collective research project (50\%).

\hypertarget{research-project}{%
\subsubsection{Research project}\label{research-project}}

The report of the research project involves the collection, examination,
and analysis of economic network data followed by a short essay
reviewing empirical and theoretical arguments. Groups of 2-3 students
will focus on a specific economic question and use network thinking and
network analysis tools to answer it. Make sure not to apply blindly the
tools you will learn in the class but rather tell a story with your
network data. You will be guided along your projects in group meetings.
The outcome of your project should be (1) a slidedeck (10 slides max),
(2) a 2000-word long paper (excluding references) and (3) the R scripts
you used to produce the network graphs and compute the network metrics.
This material should be submitted to
\href{https://www.dropbox.com/request/LwSV7hxd0i4qTatae9hZ}{this}
Dropbox folder by November 1 at the latest (week 44 is free so you can
finalize the project).

\hypertarget{group-meetings}{%
\subsubsection{Group meetings}\label{group-meetings}}

One of the most important parts of this class is your network project.
You will learn many new skills, that can lead to more advanced research
or even a business opportunity. But this project is also challenging,
from truly applying network thinking and collecting the data, to
presenting your finding in a clear and structured way. These group
meetings are milestones to make sure that you stay on schedule and do
not get lost in the project complexity.

\hypertarget{readings}{%
\subsubsection{Readings}\label{readings}}

There is no class reader. The weekly readings are provided on this
web-page and slides/videos will be regularly uploaded here. All articles
listed should be considered mandatory reading. Additional online
materials might be assigned throughout the quarter.

\hypertarget{course-schedule}{%
\subsubsection{Course Schedule}\label{course-schedule}}

For the online webinars we have created an `Economics of Networks' team
in MS Teams, you will be able to join the team by using this code:
8a4fps9.

\begin{longtable}[]{@{}cccccccc@{}}
\toprule
Week & Day & Date & Time & Format & Location & Activity &
Lecturer\tabularnewline
\midrule
\endhead
37 & Thursday & 10/09 & 13:30-15:00 & Webinar & Online & Lecture 1 &
Balland\tabularnewline
37 & Thursday & 10/09 & 15:15-17:00 & Webinar & Online & Lab 1 &
Balland\tabularnewline
38 & Thursday & 17/09 & 14:00-15:30 & Webinar & Online & Lecture 2 &
Balland\tabularnewline
38 & - & - & - & Video & Online & Lab 2 & Balland\tabularnewline
38 & Thursday & 17/09 & 15:45-17:00 & Webinar & Online & Lecture 3 &
Balland\tabularnewline
38 & - & - & - & Video & Online & Lab 3 & Balland\tabularnewline
39 & Thursday & 24/09 & 13:15-17:00 & Call & Online & Meeting & De
Bruin\tabularnewline
40 & Tuesday & 29/09 & 11:00-12:45 & Webinar & Online & Lecture 4 &
Balland\tabularnewline
40 & - & - & - & Video & Online & Lab 4 & Balland\tabularnewline
40 & Thursday & 01/10 & 13:15-17:00 & Call & Online & Meeting & De
Bruin\tabularnewline
41 & Thursday & 08/10 & 13:15-15:00 & Exam & Online & Exam & De
Bruin\tabularnewline
42 & Thursday & 15/10 & 13:15-17:00 & Call & Online & Meeting & De
Bruin\tabularnewline
43 & Thursday & 22/10 & 13:15-17:00 & Call & Online & Meeting & De
Bruin\tabularnewline
\bottomrule
\end{longtable}

\hypertarget{lecture-1-introduction-to-network-science-slides}{%
\subparagraph{\texorpdfstring{\textbf{Lecture 1: Introduction to network
science}
\href{https://github.com/PABalland/PABalland.github.io/raw/master/teaching/on/L1\&Lab1.pdf}{SLIDES}}{Lecture 1: Introduction to network science SLIDES}}\label{lecture-1-introduction-to-network-science-slides}}

\emph{Topics covered}\\
- Overview of class\\
- Introduction to network thinking\\
- Networks in natural sciences, social sciences, and business\\
- Economics and networks

\emph{References}\\
- Barabasi, A. L. (2012) The network takeover, Nature Physics 8 (1),
14-16
\href{https://2c2e773f-85c0-4039-818c-ea517fc75085.filesusr.com/ugd/c5611b_c65333e3828c4a7a854f20ff09e4b127.pdf}{PDF}\\
- Ter Wal, A. L., and Boschma, R. A. (2009) Applying social network
analysis in economic geography: framing some key analytic issues. The
Annals of Regional Science 43 (3): 739-756
\href{https://2c2e773f-85c0-4039-818c-ea517fc75085.filesusr.com/ugd/c5611b_0931e6065e214ba1933018110703fae3.pdf}{PDF}\\
- Hanneman, R.A. and Riddle, M. (2005) Introduction to social network
methods. Riverside, CA: University of California, Riverside - Chapter 1
\href{http://faculty.ucr.edu/~hanneman/nettext/C1_Social_Network_Data.html}{PDF}\\
- An application of ONA by Deloitte
\href{https://www2.deloitte.com/us/en/pages/human-capital/articles/organizational-network-analysis.html}{PDF}

\hypertarget{lab-1-network-analysis-in-r---introduction-to-r-slides}{%
\subparagraph{\texorpdfstring{\textbf{Lab 1: Network Analysis in R -
Introduction to R}
\href{https://paballand.github.io/teaching/on/Lab1.html}{SLIDES}}{Lab 1: Network Analysis in R - Introduction to R SLIDES}}\label{lab-1-network-analysis-in-r---introduction-to-r-slides}}

\emph{Topics covered}\\
- Network data collection (research projects)\\
- Software for network analysis\\
- Introduction to R and RStudio\\
- Basic programming skills

References\\
- \url{http://www.statmethods.net}\\
- \url{https://pballand.wixsite.com/balland/install-r}

\hypertarget{lecture-2-graph-theory-and-complex-networks-slides}{%
\subparagraph{\texorpdfstring{\textbf{Lecture 2: Graph theory and
Complex networks}
\href{https://github.com/PABalland/PABalland.github.io/raw/master/teaching/on/L2.pdf}{SLIDES}}{Lecture 2: Graph theory and Complex networks SLIDES}}\label{lecture-2-graph-theory-and-complex-networks-slides}}

\emph{Topics covered}\\
- Graphs and matrices\\
- Key concepts: nodes, links, structure\\
- Random networks\\
- Small worlds\\
- Growing networks\\
- Key structural patterns of real-world networks\\
- Principles to keep in mind when working with your own network data\\
- Scope and milestones of the project

\emph{References}\\
- Barabasi, A. L. (2016) Network Science. Cambridge, England: Cambridge
University Press - Chapter 2
\href{http://networksciencebook.com/chapter/2}{PDF}\\
- Hanneman, R.A. and Riddle, M. (2005) Introduction to social network
methods. Riverside, CA: University of California, Riverside - Chapters 2
\href{http://faculty.ucr.edu/~hanneman/nettext/C2_Formal_Methods.html}{PDF},
3 \href{http://faculty.ucr.edu/~hanneman/nettext/C3_Graphs.html}{PDF}, 5
\href{http://faculty.ucr.edu/~hanneman/nettext/C5_\%20Matrices.html}{PDF}
\& 7
\href{http://faculty.ucr.edu/~hanneman/nettext/C7_Connection.html}{PDF}\\
- Barabasi, A. L., and Albert, R. (1999) Emergence of scaling in random
networks, Science 286 (5439): 509-512
\href{https://2c2e773f-85c0-4039-818c-ea517fc75085.filesusr.com/ugd/c5611b_0f8a62d4615e4821859c249a3a0802af.pdf}{PDF}\\
- Watts, D. J., and Strogatz, S. H. (1998) Collective dynamics of
`small-world' networks, Nature 393 (6684): 440-442
\href{https://2c2e773f-85c0-4039-818c-ea517fc75085.filesusr.com/ugd/c5611b_a47348d589f943d38faf71c2c1769b79.pdf}{PDF}

\hypertarget{lab.-2-network-analysis-in-r---network-data}{%
\subparagraph{\texorpdfstring{\textbf{Lab. 2: Network analysis in R -
Network
Data}}{Lab. 2: Network analysis in R - Network Data}}\label{lab.-2-network-analysis-in-r---network-data}}

\emph{Topics covered}\\
- Basic matrix algebra
\href{https://www.youtube.com/watch?v=cac1yqxxwWM\&feature=youtu.be}{VIDEO
2.1}\\
- Network data management
\href{https://www.youtube.com/watch?v=TAojsuvZRsI\&feature=youtu.be}{VIDEO
2.2}\\
- Creating an igraph object (from raw data)
\href{https://www.youtube.com/watch?v=gRkCgIejXCI}{VIDEO 2.3}

\emph{References}\\
- Balland, P.A. (2017) Economic Geography in R: Introduction to the
EconGeo Package, Papers in Evolutionary Economic Geography, 17 (09):
1-75
\href{https://peeg.wordpress.com/2017/05/09/17-09-economic-geography-in-r-introduction-to-the-econgeo-package/}{PDF}\\
- Csardi G, Nepusz T: The igraph software package for complex network
research, InterJournal, Complex Systems 1695. 2006
\href{http://www.necsi.edu/events/iccs6/papers/c1602a3c126ba822d0bc4293371c.pdf}{PDF}

\hypertarget{lecture-3-centrality-and-power-slides}{%
\subparagraph{\texorpdfstring{\textbf{Lecture 3: Centrality and power}
\href{https://github.com/PABalland/PABalland.github.io/raw/master/teaching/on/L3.pdf}{SLIDES}}{Lecture 3: Centrality and power SLIDES}}\label{lecture-3-centrality-and-power-slides}}

\emph{Topics covered}\\
- Why positions of actors/nodes in a network matter\\
- Degree, betweenness, and closeness centrality\\
- Strength of weak ties, structural holes, and network closure

\emph{References}\\
- Granovetter, M. (1985) Economic action and social structure: The
problem of embeddedness, American journal of sociology 91 (3): 481-510
\href{https://2c2e773f-85c0-4039-818c-ea517fc75085.filesusr.com/ugd/c5611b_0c0ac97925d74aaa914141a15b3bd734.pdf}{PDF}\\
- Burt, R. S. (2004) Structural holes and good ideas. American journal
of sociology 110 (2): 349-399
\href{http://www.econ.upf.edu/docs/seminars/burt.pdf}{PDF}\\
- Hanneman, R.A. and Riddle, M. (2005) Introduction to social network
methods. Riverside, CA: University of California, Riverside - Chapters 9
\href{http://faculty.ucr.edu/~hanneman/nettext/C9_Ego_networks.html}{PDF}
\& 10
\href{http://faculty.ucr.edu/~hanneman/nettext/C10_Centrality.html}{PDF}

\hypertarget{lab.-3-network-analysis-in-r---computing-global-local-indicators}{%
\subparagraph{\texorpdfstring{\textbf{Lab. 3: Network analysis in R -
Computing global \& local
indicators}}{Lab. 3: Network analysis in R - Computing global \& local indicators}}\label{lab.-3-network-analysis-in-r---computing-global-local-indicators}}

\emph{Topics covered}\\
- Structural analysis of global networks
\href{https://www.youtube.com/watch?v=JI3ibVmeopU\&feature=youtu.be}{VIDEO
3.1}\\
- Computing different forms of centrality
\href{https://youtu.be/57kuxJ-dVAI}{VIDEO 3.2}\\
- Mastering the igraph R package

\hypertarget{lecture-4-the-economy-as-a-complex-system-slides}{%
\subparagraph{\texorpdfstring{\textbf{Lecture 4: The economy as a
complex system}
\href{https://github.com/PABalland/PABalland.github.io/raw/master/teaching/on/L4.pdf}{SLIDES}}{Lecture 4: The economy as a complex system SLIDES}}\label{lecture-4-the-economy-as-a-complex-system-slides}}

\emph{Topics covered}\\
- Endogenous mechanisms of growth in economic systems\\
- Mapping economic systems as 2-mode networks\\
- Relatedness and evolution\\
- Predicting changes in economic systems\\
- An application to European innovation policy

\emph{References}\\
- Hidalgo, C., Balland, P.A., Boschma, R., Delgado, M., Feldman, M.,
Frenken, K., Glaeser, E., He, C., Kogler, D., Morrison, A., Neffke, F.,
Rigby, D., Stern, S., Zheng, S., and Zhu, S. (2018) The Principle of
Relatedness, Proceedings of the 20th International Conference on Complex
Systems, forthcoming {[}PDF{]}\\
- Hidalgo, C. A., Klinger, B., Barabasi, A. L., \& Hausmann, R. (2007).
The product space conditions the development of nations. Science,
317(5837), 482-487 \href{http://barabasi.com/f/220.pdf}{PDF}\\
- Balland, P.A., Boschma, R., Crespo, J. and Rigby, D. (2019) Smart
Specialization policy in the EU: Relatedness, Knowledge Complexity and
Regional Diversification, Regional Studies, forthcoming
\href{https://2c2e773f-85c0-4039-818c-ea517fc75085.filesusr.com/ugd/c5611b_94433b170c40428d9fea06a58dcef6d1.pdf}{PDF}\\
- INET Webinar \href{https://www.youtube.com/watch?v=BmMTvj6IWRk}{How
Regions Can Re-invent Themselves} by Pierre-Alexandre Balland

\hypertarget{lab.-4-mapping-the-structure-of-economic-systems-in-r}{%
\subparagraph{\texorpdfstring{\textbf{Lab. 4: Mapping the structure of
economic systems in
R}}{Lab. 4: Mapping the structure of economic systems in R}}\label{lab.-4-mapping-the-structure-of-economic-systems-in-r}}

\emph{Topics covered}\\
- Computing relatedness between (economic) activities
\href{https://youtu.be/l5T0lGTQfWw}{VIDEO 4.1}\\
- Relatedness density and predicting entry (diversification)
\href{https://youtu.be/1G5bbk8ZXDA}{VIDEO 4.2}\\
- Visualization of complex networks
\href{https://youtu.be/rRK0o9GRCsg}{VIDEO 4.3}

\textbf{Additional reading}\\
Students are not required to purchase any books to follow this course.
If you are interested in additional reading, these three books make an
excellent introduction to the world of network science and network
analysis:

\begin{itemize}
\tightlist
\item
  Barabasi, A.L. (2002) Linked: The New Science of Networks. Cambridge,
  MA: Perseus.\\
\item
  Newman, M.E.J. (2010) Networks: An Introduction. Oxford, England:
  Oxford University Press.\\
\item
  Wasserman, S., and Faust, K. (1994) Social Network Analysis: Methods
  and Applications. Cambridge, England: Cambridge University Press.
\end{itemize}

\textbf{Useful websites}

\begin{itemize}
\tightlist
\item
  Full introductory online textbook on social network analysis by Robert
  Hanneman and Mark Riddle:
  \url{http://faculty.ucr.edu/~hanneman/nettext/}\\
\item
  Full introductory online textbook on network science by Albert-L?szl?
  Barab?si:
  \url{http://barabasilab.neu.edu/networksciencebook/downlPDF.html}\\
\item
  A simple introduction to R by Robert Kabacoff:
  \url{http://www.statmethods.net/}\\
\item
  The website of Tom Snijders on dynamic network analysis:
  \url{https://www.stats.ox.ac.uk/~snijders/siena/}
\end{itemize}

\end{document}
